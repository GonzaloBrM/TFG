\documentclass[a4paper,12pt]{article}
\begin{document}

\title{Análisis y aplicación de modelos de aprendizaje computacional al campo de los videojuegos }
\author{Gonzalo Bracamonte Martínez}
\date{\today}
\maketitle



\title{Conceptos básicos de aprendizaje computacional}

Este trabajo tiene como objetivo realizar un análisis que permita concluir que campos del Machine Learning son aplicables a qué campos de los videojuegos. Es decir, buscamos comprender qué técnicas y modelos permiten a componentes de este contexto aprender de un conjunto de datos sin ser explícitamente programados. De este modo, el desarrollo de dicho análisis deberá tener en cuenta los 3 elementos principales del aprendizaje: Por un lado trataremos de identificar que tipos de experiencia se asocian a cada uno de los entornos de estudio y cuales son los modelos aplicables para dicha experiencia. Para determinar el modelo también requerimos de un análisis del tipo de tareas a ejecutar por el algoritmo y una medida de rendimiento de la misma. Estos tres parámetros nos sirven como punto de partida para la clasificación de los problemas, por el otro lado, deberemos determinar, de la forma mas genérica y abstracta posible, cuales son las carácteristicas principales del campo de estudio que van a influir en la decisión de modelo computacional.

\title{Campos de estudio}


\end{document}